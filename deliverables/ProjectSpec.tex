\documentclass[12pt]{article}
\usepackage[utf8]{inputenc}
\usepackage[T1]{fontenc}
\usepackage{hyperref}
\usepackage{lipsum} % for placeholder text, remove if not needed
\usepackage[a4paper, portrait, margin=1.5cm]{geometry}
\usepackage{graphicx}

\title{IBM AI AR techdoc}
\author{Ani Bitri}

\begin{document}

\maketitle

\tableofcontents
\newpage

\section{Problem Statement}

    \subsection{Introduction and Context}

        Technical documentation plays a critical role in helping developers understand, implement, operate and maintain software systems. In enterprise environments, documentation is the primary interface between developers and complex
        systems, making clarity and accessibility essential. However, despite its importance, technical documentation remains predominantly static, text-based and difficult to navigate. As systems grow in complexity, dynamic processes such
        as data flows, dependencies and interactions are often poorly represented in traditional documentation formats. Consequently, many users often struggle to find the information they need, leading to frustration, errors and inefficiencies. This 
        raises the need for innovative solutions that can enhance the user experience and improve comprehension of technical documentation.

    \subsection{Existing Technologies}



    \subsection{Gaps in Current Solutions}

    \subsection{Problem Definition}


\section{Project Overview}

    \subsection{Purpose}
    Many developers face challenges while reading and understanding technical documentation. This project aims to create an AI-powered mobile application which, supported by AR technology, will enhance the user experience
    by providing interactive and immersive documentation. Powered by IBM Watson/Grantie, the app will offer several features to assist developers in navigating and comprehending complex technical documents, including text recognition,
    interactive AR overlays, chatbot assistance and more.

    \subsection{Objectives within the scope}
    The primary objectives of this project are to:
    \begin{enumerate}
        \item Implement AR diagram augmentation
        \begin{itemize}
            \item Detect and track diagrams in printed and digital forms.
            \item Overlay interactive elements on diagrams to provide additional context and explanations.
        \end{itemize}
        \item Integrate an AI assistant
        \begin{itemize}
            \item Employ IBM Watson/Granite to interpret the scanned documentation text and answer user queries.
            \item Support natural lanugage questions such as "What is the purpose of this diagram?" or "Explain this concept in simpler terms.".
        \end{itemize}
        \item Preserve accessability and compliance
        \begin{itemize}
            \item Keep the core document unchanged and externalize enhancements through AR overlays to comply with accessibility and legal requirements.
        \end{itemize}
        \item Develop a functional mobile prototype
        \begin{itemize}
            \item Deliver a working mobile application prototype that demonstrates the key features and functionalities.
            \item Conduct user testing to gather feedback and refine the application.
            \item Provide a short demonstration.
        \end{itemize}
        \end{enumerate}

    \subsection{Boundaries and Out-of-Scope Elements}
    To keep the project achievable within the given timeframe, the following elements are considered out of scope:
    \begin{itemize}
        \item Full production deployment or enterprise-level integration.
        \item Hardware-specific AR is excluded; the focus is on mobile devices.
        \item Cross-platform optimization beyond the primary target platform (e.g., iOS or Android).
    \end{itemize}

    \subsection{Expected Deliverables}
    \begin{itemize}
        \item A prototype mobile application demonstrating real-time recognition and overlay of technical documentation.
        \item Integrated AI assistant interface capable of answering user queries based on the documentation content.
        \item A comprehensive project report detailing the design, implementation, testing processes and evaluation results.
        \item VIVA presentation and demonstration.
    \end{itemize}

    \subsection{Target Outcomes}
    \begin{itemize}
        \item Enhanced user experience for developers interacting with technical documentation.
        \item Improved comprehension of complex technical concepts through interactive AR elements and AI assistance.
        \item A foundation for future development and potential commercialization of the application.
    \end{itemize}

\section{Requirements}
Each requirement will be described in the format R{n}C/D, where n is the requirement number and C or D indicates whether the requirement is customer (C) or developer (D) oriented. For example,
R1C refers to the first customer requirement, while R2D refers to the second developer requirement. Each requirement will be detailed with its description, priority, verification method and traceability.

    \subsection{Functional Requirements}
    List the key functionalities the software system must support. Provide clear and concise descriptions of features and interactions.

    \begin{enumerate}
        \item \textbf{Augmented Reality System}
    \end{enumerate}

    \subsection{Non-Functional Requirements}
    Outline performance, usability, reliability, and other quality attributes expected from the system.

\section{System Architecture}
Provide a high-level overview of the system architecture. Include diagrams where appropriate to illustrate the system components and their interactions.

    \subsection{Component Diagram}
    \begin{center}
    % Insert component diagram here
    \end{center}

    \subsection{Frontend Design}

    \subsection{Backend Design}

    \subsection{Technology Stack}
    List the technologies, programming languages, frameworks, and tools that will be used in the project.

\section{Development Plan and Project Philosophy}

    \subsection{Methodology}
    Describe the development methodology (e.g., Agile, Waterfall) that will be followed during the project lifecycle.

    The project will follow the Scrum Agile methodology.

    \subsection{Time Management and Milestones}
    Detail the project timeline with key milestones and deliverables.

    \subsection{Resource Management}
    Outline the resources (e.g., personnel, equipment) required for the project and how they will be allocated.

    \subsection{Risk Management}
    Identify potential risks and outline strategies for mitigating them.

\section{Conclusion}
Summarize the key points of the specification document and outline next steps.

\end{document}